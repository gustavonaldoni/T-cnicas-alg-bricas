\documentclass[14pt, aspectratio=169]{beamer}

\usetheme{Copenhagen}
\setbeamertemplate{navigation symbols}{} % Esconde as barras de navegação inferiores
\setbeamercovered{transparent}
\setbeamertemplate{headline}{} % Esconde a barra de navegação no topo do documento, relativa às seções

\newcommand{\C}{\mathbb{C}}
\newcommand{\R}{\mathbb{R}}
\newcommand{\I}{\mathbb{I}}
\newcommand{\Q}{\mathbb{Q}}
\newcommand{\Z}{\mathbb{Z}}
\newcommand{\N}{\mathbb{N}}

\newcommand{\conj}[1]{\left\{ #1 \right\}}
\newcommand{\skipframe}{\vspace{10.0cm}}
\newcommand{\parenthesis}[1]{\left( #1 \right)}

\newtheorem{theo}{Teorema}
\newtheorem{ex}{Exemplo}

% Pacotes Essenciais ========================================================
\usepackage[utf8]{inputenc} % Acentuação
\usepackage[T1]{fontenc} % Fontes
\usepackage{graphicx,xcolor} % Gráficos e Cores
\usepackage{enumerate} % Enumeração
\usepackage{nameref} % Referência por nome
\usepackage{amsfonts}
\usepackage{amsmath}
\usepackage{amssymb}
\usepackage{amsthm}
\usepackage{hyperref}
\usepackage{courier}
\usepackage[brazil]{babel}

\usepackage[alf, abnt-repeated-title-omit=yes, abnt-emphasize=bf, abnt-etal-list=0]{abntex2cite}
\citebrackets()

\usepackage{multicol,multirow} % Multicolunas e multitabelas

\usepackage{listings}
\usepackage{xcolor}

\definecolor{codegreen}{rgb}{0,0.6,0}
\definecolor{codegray}{rgb}{0.5,0.5,0.5}
\definecolor{codepurple}{rgb}{0.58,0,0.82}
\definecolor{backcolour}{rgb}{0.95,0.95,0.92}

\lstdefinestyle{mystyle}{
    backgroundcolor=\color{backcolour},   
    commentstyle=\color{codegreen},
    keywordstyle=\color{magenta},
    numberstyle=\tiny\color{codegray},
    stringstyle=\color{codepurple},
    basicstyle=\ttfamily\footnotesize,
    breakatwhitespace=false,         
    breaklines=true,                 
    captionpos=b,                    
    keepspaces=true,                 
    numbers=left,                    
    numbersep=5pt,                  
    showspaces=false,                
    showstringspaces=false,
    showtabs=false,                  
    tabsize=2
}

\lstset{style=mystyle}

\usepackage{tikz}
\usepackage{booktabs}
\usepackage{polynom}


\title{Técnicas algébricas}
\subtitle{Pré-cálculo}
\author{Prof. Dr. Márcio Leandro Gonçalves}
\date{\today}
\institute{PUC Minas - Poços de Caldas}

\begin{document}

\begin{frame}
\maketitle 
\end{frame}

\begin{frame}{Sumário}
    \tableofcontents
\end{frame}

\section{Polinômios}

\begin{frame}[allowframebreaks]{Polinômios}

\begin{itemize}
    \item Uma função $f: \C \rightarrow \C$ é chamada de \emph{função polinomial} ou \emph{polinômio} de grau $n$ desde que a condição $f(x) = a_1 x^0 + a_2 x^1 + a_3 x^2 + \cdots + a_n x^n$ seja satisfeita para $a_1, a_2, a_3, \ldots, a_n \in \C$, $n \in \N$ e $a_n \neq 0$.

    \item Os números $a_1, a_2, a_3, \ldots, a_n$ são chamados de \emph{coeficientes} de $f$, enquanto $a_1 x^0, a_2 x^1, a_3 x^2, \cdots, a^n x^n$ são os seus \emph{termos} ou \emph{monômios}.

    \skipframe

    \item Exemplos:
    \begin{itemize}
        \item $p_1(x) = x^2 + 3x + 1$
        \item $p_2(x) = \sqrt{2}x^8 - 5x^3 + \dfrac{x^2}{2}$
        \item $p_3(x) = 4x^{10} + 3x^5 + 9x$
    \end{itemize}

    \skipframe

    \item Uma função polinomial com um único termo é chamada de \emph{função monomial} ou \emph{monômio}.

    \item Exemplos:
    \begin{itemize}
        \item $p_1(x) = 3x^5$
        \item $p_2(x) = \dfrac{x^6}{\sqrt{2}}$
        \item $p_3(x) = 2$
    \end{itemize}

    \skipframe

    \item Um polinômio $p(x)$ \emph{nulo} é aquele que, para todo $x \in \C$, $p(x) = 0$. Matematicamente $p = 0 \Leftrightarrow p(x) = 0 \mid \forall x \in \C$.

    \item Note que, se $p = 0$, então $a_1 = a_2 = a_3 = \cdots = a_n = 0$. Em outras palavras, num polinômio nulo é necessário que todos os seus coeficientes também sejam nulos.

    \skipframe

    \item Dadas duas funções polinomiais $f_1(x)$ e $f_2(x)$, se $f_1(x)$ for idêntica a $f_2(x)$, então, para todo $x \in \C$, $f_1(x) = f_2(x)$.

    \item Note que, se $f_1$ é idêntico a $f_2$, então seus coeficientes devem ser ordenadamente iguais.

    \skipframe

    \item As raízes de um polinômio $p$ de grau $n$ são $r_1, r_2, r_3, \ldots, r_n \in \C$ de tal modo que $p(r_1) = p(r_2) = p(r_3) = \cdots = p(r_n) = 0$.

    \item Em outras palavras, as raízes de $p$ são números complexos que, ao substituírem $x$ em $p(x)$, retornam uma imagem nula.
\end{itemize}
    
\end{frame}

\section{Fatoração de polinômios}

\begin{frame}[allowframebreaks]{Fatoração de polinômios}

\begin{itemize}

    \item Fatorar um polinômio significa escrevê-lo como uma multiplicação de dois ou mais polinômios.

    \item A seguir serão mostrados os principais métodos usados na hora de fatorar um polinômio qualquer.

    \skipframe

    \item Fator comum em evidência: consiste em colocar um fator comum de dois monômios em evidência, pela propriedade distributiva da multiplicação.
    \begin{equation*}
        ax + bx = x(a + b)
    \end{equation*}

    \item Exemplos:
    \begin{itemize}
        \item $4x + 20 = 4(x+5)$
        \item $3xy + 9xz + 6x = 3x(y + 3z + 2)$
        \item $-4yx + 2xyz = 2y(-2x + xz)$
    \end{itemize}

    \skipframe

    \item Agrupamento: consiste em separar a expressão em dois grupos e colocar em evidência o fator comum de cada um deles.
    \begin{equation*}
        ax + ay + bx + by = a(x+y) + b(x+y) = (x+y)(a+b)
    \end{equation*}

    \item Exemplos:
    \begin{itemize}
        \item $x^2 - ax + xy - ay = x(x-a) + y(x-a) = (x-a)(x+y)$
        \item $8ax + bx + 8ay + by = x(8a+b) + y(8a+b) = (8a + b)(x+y)$
    \end{itemize}

    \skipframe

    \item Teorema da decomposição: dado um polinômio $p$ de grau $n$ de raízes complexas $r_1, r_2, \cdots, r_n$, pode-se fatorá-lo no formato $p(x) = a_n(x - r_1)(x - r_2) \cdots (x - r_n)$, em que $a_n$ é o coeficiente do termo $a_n x^n$.

    \skipframe

    \item De forma prática, para um polinômio $p$ de grau 2, pode-se escrevê-lo no formato $p(x) = a(x - r_1)(x - r_2)$, em que $a$ é o coeficiente do termo $ax^2$.

    \item Exemplos:
    \begin{itemize}
        \item $p_1(x) = 3x^2 + 24x + 36 = 3(x + 6)(x + 2)$
        \item $p_2(x) = 2x^2 - 6x - 20 = 2(x - 5)(x + 2)$
    \end{itemize}

\end{itemize}
    
\end{frame}

\section{Produtos notáveis}

\begin{frame}{Produtos notáveis}

\begin{itemize}
    \item $(a + b)^2 = a^2 + 2ab + b^2$
    \item $(a - b)^2 = a^2 - 2ab + b^2$
    \item $a^2 - b^2 = (a + b)(a - b)$
    \item $(a + b)^3 = a^3 + 3a^2b + 3ab^2 + b^3$
    \item $(a - b)^3 = a^3 - 3a^2b + 3ab^2 - b^3$
    \item $a^3 + b^3 = (a + b)(a^2 - ab + b^2)$
    \item $a^3 - b^3 = (a - b)(a^2 + ab + b^2)$
\end{itemize}
    
\end{frame}

\section{Divisão de polinômios}

\section{Frações algébricas}

\begin{frame}[allowframebreaks]{Frações algébricas}
    
\end{frame}

\section{Racionalização}

\begin{frame}[allowframebreaks]{Racionalização}

\begin{itemize}
    \item Exemplos:
    \begin{itemize}
        \item $\dfrac{2}{\sqrt{3}} = \dfrac{2}{\sqrt{3}} \cdot \dfrac{\sqrt{3}}{\sqrt{3}} = \dfrac{2\sqrt{3}}{3}$
    \end{itemize}
    
\end{itemize}
    
\end{frame}

\section{Exercícios}

\begin{frame}[allowframebreaks]{Exercícios}
    \begin{enumerate}
        \item Simplifique as expressões abaixo utilizando técnicas algébricas.
        
        \begin{multicols}{2}
            \begin{enumerate}[a]
                \item $\dfrac{x^2 + x - 6}{x - 2}$
                \item $\dfrac{t^2 - 9}{2t^2 + 7t + 3}$
                \item $\dfrac{\parenthesis{4+h}^2 - 16}{h}$
                \item $\dfrac{x+2}{x^3 + 8}$
                \item $\dfrac{9 - t}{3 - \sqrt{t}}$
                \item $\dfrac{\sqrt{x+2}-3}{x-7}$   
                \item $\dfrac{\frac{1}{4}+\frac{1}{x}}{4+x}$
                \item $\dfrac{x^2-81}{\sqrt{x}-3}$
            \end{enumerate}
        \end{multicols}

        \item Simplifique cada uma das expressões abaixo.
        
        \begin{multicols}{2}
            \begin{enumerate}[a]
                \item $\dfrac{x^2 - 7x + 10}{x^2 - 4}$
                \item $\dfrac{5-\sqrt{x}}{25-x}$
                \item $\dfrac{x^2 + x - 2}{x^2 - 1}$
                \item $\dfrac{(x+3)^3 - 27}{x}$
                \item $\dfrac{x^2}{\sqrt{x^2 + 12}-\sqrt{12}}$
                \item $\dfrac{3}{x} \parenthesis{\dfrac{1}{5+x} - \dfrac{1}{5-x}}$
                \item $\dfrac{x^3 + 1}{x^2 - 1}$
                \item $\dfrac{4x^2 - 1}{2x - 1}$
            \end{enumerate}
        \end{multicols}
        
        
        
    \end{enumerate}
\end{frame}

\section{Respostas dos exercícios}

\begin{frame}[allowframebreaks]{Respostas dos exercícios}

\begin{enumerate}
    \item 

    \begin{multicols}{2}
        \begin{enumerate}[a]
            \item $x + 3$
            \item $\dfrac{t - 3}{2t + 1}$
            \item $8 + h$
            \item $\dfrac{1}{x^2 - 2x + 4}$
            \item $3 +  \sqrt{t}$
            \item $\dfrac{1}{\sqrt{x+2} + 3}$
            \item $\dfrac{1}{4x}$
            \item $\parenthesis{x+9}\parenthesis{\sqrt{x}+3}$
            \end{enumerate}
    \end{multicols}
    

    \item 

    \begin{multicols}{2}
        \begin{enumerate}[a]
            \item $\dfrac{x-5}{x+2}$
            \item $\dfrac{1}{5 + \sqrt{x}}$
            \item $\dfrac{x+2}{x+1}$
            \item $x^2 + 9x + 27$
            \item $\sqrt{x^2 + 12} + \sqrt{12}$
            \item $-\dfrac{6}{\parenthesis{5+x}\parenthesis{5-x}}$
            \item $\dfrac{x^2-x+1}{x-1}$
            \item $2x+1$
            \end{enumerate}
    \end{multicols}
\end{enumerate}
    
\end{frame}

\section{Referências bibliográficas}

\begin{frame}{Referências bibliográficas}
    \bibliography{referencias}
\end{frame}

\end{document}